%The \introduction command is provided as a convenience.
%if you want special chapter formatting, you'll probably want to avoid using it altogether
		

      \chapter*{Introduction}
         \addcontentsline{toc}{chapter}{Introduction}
	\chaptermark{Introduction}
	\markboth{Introduction}{Introduction}
	% The three lines above are to make sure that the headers are right, that the intro gets included in the table of contents, and that it doesn't get numbered 1 so that chapter one is 1.

% Double spacing: if you want to double space, or one and a half 
% space, uncomment one of the following lines. You can go back to 
% single spacing with the \singlespacing command.
% \onehalfspacing
% \doublespacing
	
	Fluid dynamics is an enormous, and relatively new branch of physics.  In addition to Physics, it's a field of interest for both Engineering and Mathematics.  Its applicability to both liquid and gas flows is useful in many Engineering problems, from aerodynamics to fluid cooling mechanisms [source!].  Fluid dynamical systems are also almost always nonlinear, making them interesting and incredibly difficult Mathematical problems.

	Technically, 2-dimensional fluid dynamics problems do not exist in nature, since nature is intrinsically 3-dimensional.  Relatively thin fluid layers, such as the earths oceans and atmosphere, or just a thin layer of fluid in the lab can be accurately approximated as 2-dimensional systems since area is so many times greater than depth.  I chose to work with a 2-dimensional thin fluid layer because it's much easier to work with than a 3-dimensional system.  Experimentally, a 2-dimensional system enables me to place and view tracer particles on the surface instead of submerged in the fluid.  Computationally, 2-dimensional systems are also far easier to model, requiring a simpler code, and  significantly less computational power.  This allowed me to write a program myself, that I could run on a regular personal computer.
	
	  The system I'm working with is called a 2-dimensional Kolmogorov flow.  This system, first studied by Soviet physicist and mathematician A. N. Kolmogorov, is defined by a sinusoidal forcing pattern in along one axis, and no forcing along the other.  This creates parallel flow channels moving in opposite directions, exerting a shear force against one another.  This shear is a result of the forcing strength and viscosity of the fluid, which are both represented in the Reynolds Number, the independent variable.
	

	New techniques for analyzing nonlinear systems paired with a few important technological developments are what made the study of fluid dynamics possible in the last fifty years.  Fluid dynamical systems are usually impossible to solve analytically, so were basically unsolvable until computers became powerful enough to provide computational solutions (source, quotes from nonlin book; something about computers in nonlin.), and experimental measuring tools were precise enough to provide an accurate model of a fluid flow.  Below I provide some historical background and basic information about both of these important developments.

\section{Some Numerical Background}

	The Navier-Stokes equations are the governing equations of fluid dynamics.  These equations are based off of Newton's laws, and conservation of mass and energy, and were derived in the first half of the nineteenth century separately by M. Navier and G. Stokes [CFD 41].  Basically it is the conservation laws and Newton's Second Law applied to an infintessimal volume of fluid.  These equations are derived in Chapter 2, Theory.
	
	Computational Fluid Dynamics became a viable way to analyze fluid systems in the mid 1960s [CFD 17].  At this time the Cold war was in full swing, and a huge amount of research was going into making good aerodynamics for both Intercontinental Ballistic Missiles and space travel technology.  Before this time, the only options were analytic solutions, which were impossible due to the on-linearity of the Navier-Stokes equations, and giant wind tunnels, which were impractical, especially to model things moving at supersonic speeds.  Numerical solutions on the other hand are limited only by the amount of information we can input into the computational model, and the amount of information a computer can process, both of which have increased enormously in the last fifty years.
	
	I wrote two different programs in \textit{Mathematica} to solve an idealized 2-dimensional Kolmogorov flow system.  The code is in Appendix A, and an explanation of the techniques used can be found in Chapter 4.


\section{Some Experimental Background}
	
	Experimental fluid dynamics faced similar issues.  The principles of flow visualization through Particle Image Velocimetry have been around since the early twentieth century, when Ludwig Prandtl observed mica particles on the surface of a flowing fluid [PIV p2].  Although this did allow him to visualize the patterns in a moving fluid, techniques for extracting quantitative data didn’t emerge until much later.  Advanced optics such as lasers and florescent particles created bright and easily imaged particles.  Better resolution and high-speed cameras allowed these images to be accurately captured.  Finally, better computer software enabled huge amounts of this data to be analyzed simultaneously.  The Von Karman Institute in Brussels first developed these techniques in the early 1980s[PIV 11-13].

\section{Format}

	This thesis has two main portions.  The first is the experimental section, where applied forcing to a thin layer of fluid and used Particle Image Velocimetry to analyze the resulting flow patterns.  The second portion is entirely computational.  I wrote a code in \textit{Mathematica} to model a two-dimensional fluid system using several different techniques.  These two portions represent very different ways to deal with a nonlinear system.  Each method has strengths and weaknesses, and together should give a more complete picture of Kolmogorov flow.

	In Chapter 2, Theory, I go over the governing equations and concepts of fluid dynamics, the method I'm using to generate turbulence in my experiment.  In Chapters 3 and 4 I go into detail about the experimental and numerical methods I used.  In 3 I explain the techniques used in Particle Image Velocimetry to map out a fluid flow, describe the specific setup I used.  In 4 I talk about code I wrote, its limitations, and how this approximates a flow.  Chapter 5 contains an analysis of the experimental data and results, as well as the numerical solutions and possible sources of error.  Finally, Chapter 6 concludes my thesis by summarizing the important results and possible suggestions to expand and improve this project.

	
